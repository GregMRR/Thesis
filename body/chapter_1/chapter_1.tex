% !TEX root = ../../thesis.tex
% dove dovresti citare qualcosa scrivi \cit 
\newpage
\phantom{}
\section{Statistics of weakly nonlinear waves}

\hl{INTRO (what is done in this chapter, write at the end), put citations in here \\
It is a brief introduction, give reference for deeper understanding, Reference history of the subject (Hassleman, Zakharov, etc)}


\subsection{Hamiltonian description of waves in continuous media}


We first turn to the construction of a general Hamiltonian method for the description of waves travelling in continuous media. In doing this
we greatly borrow from the exceptional treatment in \cite{Zakharov}. \\
\hl{Questa parte iniziale può essere più o meno ampia in base al tempo a disposizione. \\
- Hamiltoniana generica per mezzo continuo (da spazio coordinate in una scatola), notazione, commenti sulla stabilità, commenti sul fatto che possa essere
già risultato di sviluppo pertubativo (come per le onde oceaniche) e scrittura in spazio-k con 3 onde e 4 onde, simmetrie continue e discrete }\\

\hl{to the appendix: - Idea Onorato oscillatore armonico per giustificare nonrisonanza e commento su trasfomrazione canonica per eliminare termini 
non risonanti (indirizza ad appendice per conto specifico), cita sistemi fisici in cui l'eq è a 4 onde} \\

Our final Hamiltonian is 
\begin{equation}
    \Ham = \sum_k \omga_k \akstar \ak + \frac{1}{2} \sum_{k123} \Tint_{k123} \akstar \aonestar \atwo \athree \delta_{23}^{k1}.
    \label{ham4}
\end{equation}

\hl{CITA IL FATTO CHE GLI INDICI DEI k SONO SOPPRESSI}
\subsection{Resonant and non-resonant interactions}
maybe in the appendix?

\subsection{Perturbation Theory}

Often in nonlinear systems no exact solutions (or few of them) are known. We now imagine ourselves in the situation where the interaction term in our Hamiltonian 
is small enough to allow for the perturbative treatment of the equations of motion, that is the expansion of the solution in orders of some small parameter, 
and their subsequent calculation order by order. To make the expansion clearer we write an explicit $\epsilon$ factor in front of the interaction term
\footnote{The small parameter my be present as a constant in the Hamiltonian (for example the coupling $g$ in the Nonlinear Schrodinger equation) or 
it may be a placeholder for the smallness of the function $\Tint_{k123}$ in a certain subdomain of k-space (\hl{for example the interaction among gravity waves in 
the small wavenumber limit}). } . \\

Following the derivation of \cite{Onorato2020} we transform to the action-angle coordinates of the unperturbed quadratic Hamiltonian
\begin{equation}
    a_k = \sqrt{\Ia_k}e^{-i\theta_k}
    \label{action_angle}
\end{equation} 
obtaining 
\begin{equation}
    \Ham = \sum_k \omga_k \Ia_k + \frac{\epsilon}{2} \sum_{k123}\Tint_{k123}\sqrt{\Ia_k\Ia_1\Ia_2\Ia_3}e^{i(\theta_k + \theta_1 - \theta_2 - \theta_3)}\delta^{k1}_{23},
\end{equation}
by assuming that $\Tint \in \mathbb{R}$ (as is the case in a vast class of physical systems) the requirement that $\Ham \in \mathbb{R}$ implies 
\begin{equation}
    \Ham = \sum_k \omga_k \Ia_k + \frac{\epsilon}{2} \sum_{k123}\Tint_{k123}\sqrt{\Ia_k\Ia_1\Ia_2\Ia_3}\cos(\deltheta^{k1}_{34})\delta^{k1}_{23},
\end{equation}
where we defined $\deltheta^{k1}_{23}=\theta_k + \theta_1 - \theta_2 - \theta_3$. \\
We can prove that the change of coordinates \eqref{action_angle} is canonical by assuming it to be true and recovering $\ak$ and $\akstar$'s poisson 
brackets\footnote{Remembering that the true canonical variables are $\ak$ and $i\akstar$.} 
\begin{align}
    \left\{ i\akstar, \ak \right\} &= i \left( \frac{\partial \akstar}{\partial \Ia_k}\frac{\partial \ak}{\partial \theta_k}  -
    \frac{\partial \akstar}{\partial \theta_k}\frac{\partial \ak}{\partial \Ia_k}  \right) \\
    &= - i \left( \frac{i}{2} \frac{1}{\sqrt{\Ia_k}}e^{i\theta_k}\sqrt{\Ia_k}e^{-i\theta_k} + 
    \frac{i}{2} \frac{1}{\sqrt{\Ia_k}}e^{-i\theta_k}\sqrt{\Ia_k}e^{i\theta_k}\right) \\
    &= 1 .
\end{align} 

We can thus impose Hamilton equations for the new coordinates (remembering that time dependance of the coordinates is suppressed)

\begin{align}
    \dt\Ia_k &= - \frac{\partial }{\partial \theta_k}\Ham = 2 \epsilon \sum_{123} \Tint_{k123} \sqrt{\Ia_k\Ia_1\Ia_2\Ia_3} \sin(\deltheta^{k1}_{23})\delta^{k1}_{23}
    \label{Hameq1}\\
    \dt\theta_k &= \phantom{-} \frac{\partial }{\partial \Ia_k}\Ham = \omga_k + 
    \epsilon \sum_{123} \Tint_{k123} \sqrt{\frac{\Ia_1\Ia_2\Ia_3}{\Ia_k}}\cos(\deltheta^{k1}_{23})\delta^{k1}_{23}.
    \label{Hameq2}
\end{align}


Since we are essentially perturbing an infinite set of harmonical oscillators with a small interaction term, we can euristically assume that
the coordinates cannot grow indefinetly to infinity. We shall then be weary of unphysical secular terms artificially introduced by the perturbative
expansion. The Poincarè-Lindsted method allows us to remove such terms by a frequency shift
\begin{equation}
    \omga_k \rightarrow \Omga_k = \omga_k + \epsilon \left(2\sum_p \Tint_{kpkp}\Ia_p - \Tint_{kkkk}I_k \right),
\end{equation}   
togheter with a change of the summatory in $\Ham$ such that $k_2=k \hspace{2mm}\&\hspace{2mm} k_1 = k_3$, $k_3=k \hspace{2mm}\& \hspace{2mm}k_1=k_2$ and $k_1=k_2=k_3=k$ are excluded from it. \\
This particular choice is better justified in the \hl{APPENDIX WITH LINK, MAKE DUFFING EXAMPLE} or in \cite{Nazarenko2011}. \\

We may now develop perturbation theory, we start by expanding the (unknown) solutions as
\begin{align}
    \Ia_k &= \Iaz_k +  + \epsilon \Iao_k + \epsilon^2 \Iat_k + \mathcal{O}(\epsilon^3) \\
    \tha_k &= \thaz_k +  + \epsilon \thao_k + \epsilon^2 \that_k + \mathcal{O}(\epsilon^3),
\end{align}
and than substituting them into \eqref{Hameq1} and \eqref{Hameq2}. \\
We now reintroduce explicit time dependance and impose $\Iaz_k(0) = \bar{\Ia}_k$ and $\Iao_k(0) = \Iat_k(0) = 0$ to fix initial conditions on the $\Ia$s and 
$\thaz_k(0) = \bar{\tha}_k$ and $\thao_k(0) = \that_k(0) = 0$ to fix initial conditions on the $\tha$s.\\

The $\epsilon^0$ order equations are
\begin{align}
    \dt\Iaz_k &= 0   \\
    \dt\thaz_k &= \Omgaz_k, 
\end{align}
with solutions 
\begin{align}
    \Iaz_k(t) &= \bar{\Ia}_k \label{solzero1}\\
    \thaz_k(t) &= \bar{\tha}_k + \bar{\Omga}_kt, \label{solzero2}
\end{align}
where $\Omgaz_k$ and $\bar{\Omga}_k$ refer to $\Omga_k$ with only zeroeth order contribution or initial conditions respectively.\hl{ANGOLO HA TERMINE SECOLARE, 
    IN EFFETTI LO METTIAMO IN LEADING ORDER DYNAMIC PROPRIO PER QUESTO MOTIVO FORSE, 
GIUSTIFICA DOPO DICENDO CHE A NOI IMPORTA CHE AZIONE NON NE ABBIA, SICCOME ANGOLI SONO IN DEI SENI, 
E FAI NOTARE CHE ALLA FINE IL FREQUENCY SHIFT NON IFLUENZA DINAMICA, COME ATTESO ESSENDO NLO}
To be precise the $\epsilon$ terms in 
the shifted frequency should be included in the equations for $\thao$ and not $\thaz$, we however make this choice to keep the term $\Omga$ in a compact form.\\
This order reproduces the dynamics of an infinite number of integrable systems (for example decoupled harmonic oscillators), with constant actions and angles evolving linearly with time. \\

At $\epsilon$ order the equations of motion are 
\begin{align}
    \dt\Iao_k &= 2\sum_{123} \Tint_{k123} \sqrt{\Iaz_k\Iaz_1\Iaz_2\Iaz_3} \sin(\deltheta^{k1(0)}_{23})\delta^{k1}_{23}\\
    \dt\thao_k &= \sum_{123} \Tint_{k123} \sqrt{\frac{\Iaz_1\Iaz_2\Iaz_3}{\Iaz_k}}\cos(\deltheta^{k1(0)}_{23})\delta^{k1}_{23}.
\end{align}
Here the only time dependance lies in $\Delta\theta^{(0)}$ and $\Iao_k(0) = \thao_k(0) = 0$, integrating the equations gives
\begin{align}
    \Iao_k(t) &= 2\sum_{123} \Tint_{k123} \sqrt{\Iab_k\Iab_1\Iab_2\Iab_3} \frac{\delta^{k1}_{23}}{\delOmegab^{k1}_{23}}
    \left[ \cos(\delthetab^{k1}_{23}) - \cos(\delthetab^{k1}_{23} + \delOmegab^{k1}_{23}t) \right] \label{solone1}\\
    \thao_k(t) &= \sum_{123} \Tint_{k123} \sqrt{\frac{\Iab_1\Iab_2\Iab_3}{\Iab_k}} \frac{\delta^{k1}_{23}}{\delOmegab^{k1}_{23}}
    \left[ \sin(\delthetab^{k1}_{23} + \delOmegab^{k1}_{23}t) -\sin(\delthetab^{k1}_{23}) \right]. \label{solone2}
\end{align}
Where $\delOmegab$ is defined in the same fashion as $\deltheta$. \\


We should be content with this first nontrivial result, but through the sheer power of hindsight
\footnote{Developing a statistical theory of the system, the first nontrivial contribution comes from the $\epsilon^2$ order.}
we write also the 
$\epsilon^2$ order equations only for the action variables (there is no need to actually solve them). \\
Looking at \eqref{Hameq1} we seek to obtain an $\epsilon^2$ equation by substituting $\Ia$ and $\tha$ up to their $\epsilon$ order terms. By Taylor expanding the square root 
we obtain four terms of the form
\begin{equation}
    \sqrt{(\mathrm{x} +\epsilon \mathrm{y})\mathrm{\tilde{x}}} \underset{\epsilon \rightarrow 0}{\sim} 
    \sqrt{\mathrm{x}\mathrm{\tilde{x}}}\left( 1 + \frac{\epsilon \mathrm{y}}{2\mathrm{\tilde{x}}}\right),
\end{equation}
where, for example, $\mathrm{x} + \epsilon \mathrm{y} = \Iaz_k + \epsilon \Iao_k$ and $\mathrm{\tilde{x}} = \Iaz_1\Iaz_2\Iaz_3$.\\
There also appear terms of the form
\begin{equation}
    \sin(\mathrm{x} + \epsilon \mathrm{y}) \underset{\epsilon \rightarrow 0}{\sim} \sin(\mathrm{x}) + \epsilon \mathrm{y} \cos(\mathrm{x}),
\end{equation} 
where $\mathrm{x} = \deltheta^{k1(0)}_{23}$ and $\mathrm{y} = \deltheta^{k1(1)}_{23}$.\\
By plugging everything into \eqref{Hameq1} we first obtain
\begin{equation}
    \dt \Iat_k = 2 \sum_{123} \Tint_{k123} \sqrt{\Iaz_k\Iaz_1\Iaz_2\Iaz_3} \left[ \frac{1}{2}\left(\frac{\Iao_k}{\Iaz_k}+\frac{\Iao_1}{\Iaz_1}+
    \frac{\Iao_2}{\Iaz_2}+\frac{\Iao_3}{\Iaz_3} \right)\sin(\deltheta^{k1(0)}_{23})\delta^{k1}_{23} + 
    \deltheta^{k1(1)}_{23}\cos(\deltheta^{k1(0)}_{23}) \delta^{k1}_{23}\right],
\end{equation}
and then by using \eqref{solzero1}, \eqref{solzero2}, \eqref{solone1}, \eqref{solone2} and basic trigonometry we find
\begin{multline}
    \dt \Iat_k = 2 \sum_{123456}\Tint_{k123} \sqrt{\Iab_k\Iab_1\Iab_2\Iab_3\Iab_4\Iab_5\Iab_6} 
    \sum_{i=1}^{4}\frac{\Tint_{k123}\Tint_{i456}}{\sqrt{\Iab_i}\delOmegab^{i4}_{56}} \\
    \times\left(\sin(\delthetab^{k1}_{23} + \delOmegab^{k1}_{23}t - \sigma_i \delthetab^{i4}_{56}) 
    + \sin(\sigma_i \delthetab^{i4}_{56} + \sigma_i\delOmegab^{i4}_{56}t -\delthetab^{k1}_{23} - \delOmegab^{k1}_{23}t)  \right)\delta^{k1}_{23}\delta^{i4}_{56}. 
    \label{eqeps2}
\end{multline}

\subsection{Statistical description}
Having approximated the solutions to order $\epsilon$ we found ourselves with the problem of gathering 
initial conditions in infinite dimensonal systems
\footnote{Let us think of the ocean surface for example, measuring its height at a generic instant would be unfeasible.}
, we shall then renounce the deterministic approach in favour of 
a probabilistic one.\\
In general such idea is realized thorugh averaging over infinitely many realizations of the equations of motion with different initial conditions, to then extract 
average quantities more easily confrontable with experiment. In a nonlinear problem this is again highly non trivial, to simplify the endeavor we assume that
a large number of waves is present in the system, in the sense that each mode in Fourier space is highly excited. It is then reasonable to assume 
the initial phases to be uniformly distributed in the $\left[0,2\pi\right]$ segment 
\footnote{Unless the original equation of motions are known to have solitonic solutions in a certain regime of k-space, in such case phases could be correlated and not 
uniformly distributed anymore.}
. We define the averaging as
\begin{equation}
    \left\langle f(\bar{\tha}_1 \dots \bar{\tha}_N) \right\rangle_{\thab} = 
    \int_{0}^{2\pi} P(\bar{\tha}_1 \dots \bar{\tha}_N)f(\bar{\tha}_1 \dots \bar{\tha}_N) d\bar{\tha}_1 \dots d\bar{\tha}_N 
    \hspace{3mm} \text{with} \hspace{3mm}
    P(\bar{\tha}_1 \dots \bar{\tha}_N) = \frac{1}{2\pi^{N}}
\end{equation}

Looking back at the Hamiltonian \eqref{ham4} we see that the phases do not contribute to physical quantities like the energy or the wave number, 
it is in the action variables that those observables are encoded. We have now a clear plan, to find a kinetic equation, independent of initial conditions, for
the action variables. \\
THe main objective is then
\begin{equation}
    \left\langle \dt \Ia_k \right\rangle_{\thab} = \dt \left\langle \Ia_k \right\rangle_{\thab} = 
    \epsilon\left\langle \dt \Iao_k \right\rangle_{\thab} + \epsilon^2\left\langle \dt \Iat_k \right\rangle_{\thab} + 
\end{equation}



\hl{-Definizione nk(t), Derivazione alla onorato della 4-WKE, limite al continuo, commenti su assunzioni!, \\
indirizza all'Appendice per metodo Poincarè lindsted. \\
-Derivazione alla Falkovich della wke, or referral to appendix}\\
\subsection{Some properties of the kinetic equation}

\hl{-Proprietà WKE :Irreversibilità, dimostrazione del teorema H, leggi di conservazione e equazioni di continuità}

\subsection{Equilibrium stationary state}

-Equilibrium solutions and comments 

\subsection{Out of equilibrium stationary states}

-Out of Equilibrium solutions and comments, argomento fjortoft, argomento zakharov, misto dei due con commenti su forzante e dissipazione 
ed esempi fisici, ragionamenti dimensionali su forme approssimate di lambda e omega e forme generiche di esponenti, definizioni costanti di KZ e commenti su cnvergenza e 
trasformata zakh un pò di foto di cascate per sistemi fisici NON MMT

\subsection{Convergence issues (MAYBE)}