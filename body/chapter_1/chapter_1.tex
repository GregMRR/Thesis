% !TEX root = ../../thesis.tex
% dove dovresti citare qualcosa scrivi \cit 
\newpage
\phantom{}
\section{Statistics of weakly nonlinear waves}

\hl{INTRO (what is done in this chapter, write at the end), put citations in here \\
It is a brief introduction, give reference for deeper understanding, Reference history of the subject (Hassleman, Zakharov, etc)}


\subsection{Hamiltonian description of waves in continuous media}


We first turn to the construction of a general Hamiltonian method for the description of waves travelling in continuous media. In doing this
we greatly borrow from the exceptional treatment in \cite{Zakharov}. \\
\hl{Questa parte iniziale può essere più o meno ampia in base al tempo a disposizione. \\
- Hamiltoniana generica per mezzo continuo (da spazio coordinate in una scatola), notazione, commenti sulla stabilità, commenti sul fatto che possa essere
già risultato di sviluppo pertubativo (come per le onde oceaniche) e scrittura in spazio-k con 3 onde e 4 onde, simmetrie continue e discrete }\\

\hl{to the appendix: - Idea Onorato oscillatore armonico per giustificare nonrisonanza e commento su trasfomrazione canonica per eliminare termini 
non risonanti (indirizza ad appendice per conto specifico), cita sistemi fisici in cui l'eq è a 4 onde
parti da spazio fourier e analizza bene dimensionalmente la hamiltoniana e le sue componenti} \\

Our final Hamiltonian is 
\begin{equation}
    \Ham = \sum_k \omga_k \akstar \ak + \frac{1}{2} \sum_{k123} \Tint_{k123} \akstar \aonestar \atwo \athree \delta_{23}^{k1}.
    \label{ham4}
\end{equation}

\hl{CITA IL FATTO CHE GLI INDICI DEI k SONO SOPPRESSI}
\subsection{Resonant and non-resonant interactions}
maybe in the appendix?

\subsection{Perturbation Theory}

Often in nonlinear systems no exact solutions (or few of them) are known. We now imagine ourselves in the situation where the interaction term in our Hamiltonian 
is small enough to allow for the perturbative treatment of the equations of motion, that is the expansion of the solution in orders of some small parameter, 
and their subsequent calculation order by order. From a physical viewpoint the smalness of the interaction term corresponds to a 
separation of the fast time scale on which the linear term operates from the slow time scale of the nonlinear one.\\
 To make the expansion clearer we write an explicit $\epsilon$ factor in front of the interaction term
\footnote{The small parameter my be present as a constant in the Hamiltonian (for example the coupling $g$ in the Nonlinear Schrodinger equation) or 
it may be a placeholder for the smallness of the function $\Tint_{k123}$ in a certain subdomain of k-space (\hl{for example the interaction among gravity waves in 
the small wavenumber limit}). } . \\

Following the derivation of \cite{Onorato2020} we transform to the action-angle coordinates of the unperturbed quadratic Hamiltonian
\begin{equation}
    a_k = \sqrt{\Ia_k}e^{-i\theta_k}
    \label{action_angle}
\end{equation} 
obtaining 
\begin{equation}
    \Ham = \sum_k \omga_k \Ia_k + \frac{\epsilon}{2} \sum_{k123}\Tint_{k123}\sqrt{\Ia_k\Ia_1\Ia_2\Ia_3}e^{i(\theta_k + \theta_1 - \theta_2 - \theta_3)}\delta^{k1}_{23},
\end{equation}
by assuming that $\Tint \in \mathbb{R}$ (as is the case in a vast class of physical systems), $\Ham \in \mathbb{R}$ implies 
\begin{equation}
    \Ham = \sum_k \omga_k \Ia_k + \frac{\epsilon}{2} \sum_{k123}\Tint_{k123}\sqrt{\Ia_k\Ia_1\Ia_2\Ia_3}\cos(\deltheta^{k1}_{34})\delta^{k1}_{23},
\end{equation}
where we defined $\deltheta^{k1}_{23}=\theta_k + \theta_1 - \theta_2 - \theta_3$. \\
We can prove that the change of coordinates \eqref{action_angle} is canonical by assuming it to be true and recovering $\ak$ and $\akstar$'s poisson 
brackets\footnote{Remembering that the true canonical variables are $\ak$ and $i\akstar$.} 
\begin{align}
    \left\{ i\akstar, \ak \right\} &= i \left( \frac{\partial \akstar}{\partial \Ia_k}\frac{\partial \ak}{\partial \theta_k}  -
    \frac{\partial \akstar}{\partial \theta_k}\frac{\partial \ak}{\partial \Ia_k}  \right) \\
    &= - i \left( \frac{i}{2} \frac{1}{\sqrt{\Ia_k}}e^{i\theta_k}\sqrt{\Ia_k}e^{-i\theta_k} + 
    \frac{i}{2} \frac{1}{\sqrt{\Ia_k}}e^{-i\theta_k}\sqrt{\Ia_k}e^{i\theta_k}\right) \\
    &= 1 .
\end{align} 

We can thus impose Hamilton equations for the new coordinates (remembering that time dependance of the coordinates is suppressed)

\begin{align}
    \dt\Ia_k &= - \frac{\partial }{\partial \theta_k}\Ham = 2 \epsilon \sum_{123} \Tint_{k123} \sqrt{\Ia_k\Ia_1\Ia_2\Ia_3} \sin(\deltheta^{k1}_{23})\delta^{k1}_{23}
    \label{Hameq1}\\
    \dt\theta_k &= \phantom{-} \frac{\partial }{\partial \Ia_k}\Ham = \omga_k + 
    \epsilon \sum_{123} \Tint_{k123} \sqrt{\frac{\Ia_1\Ia_2\Ia_3}{\Ia_k}}\cos(\deltheta^{k1}_{23})\delta^{k1}_{23}.
    \label{Hameq2}
\end{align}


Since we are essentially perturbing an infinite set of harmonical oscillators with a small interaction term, we can euristically assume that
the coordinates cannot grow indefinetly to infinity. We shall then be weary of unphysical secular terms artificially introduced by the perturbative
expansion. The Poincarè-Lindsted method allows us to remove such terms by a frequency shift
\begin{equation}
    \omga_k \rightarrow \Omga_k = \omga_k + \epsilon \left(2\sum_p \Tint_{kpkp}\Ia_p - \Tint_{kkkk}I_k \right),
    \label{shift_eq}
\end{equation}   
togheter with a change of the summatory in $\Ham$ such that the trivial interactions\footnote{
    we call them trivial as they do not correspond to a net exchange of energy/action among different Fourier modes
} $k_2=k \hspace{2mm}\&\hspace{2mm} k_1 = k_3$, $k_3=k \hspace{2mm}\& \hspace{2mm}k_1=k_2$ and $k_1=k_2=k_3=k$ are excluded from it. \\
This particular choice is better justified in the \hl{APPENDIX WITH LINK, MAKE DUFFING EXAMPLE} or in \cite{Nazarenko2011}. \\

We may now develop perturbation theory, we start by expanding the (unknown) solutions as
\begin{align}
    \Ia_k &= \Iaz_k +  + \epsilon \Iao_k + \epsilon^2 \Iat_k + \mathcal{O}(\epsilon^3) \\
    \tha_k &= \thaz_k +  + \epsilon \thao_k + \epsilon^2 \that_k + \mathcal{O}(\epsilon^3),
\end{align}
and than substituting them into \eqref{Hameq1} and \eqref{Hameq2}. \\
We now reintroduce explicit time dependance and impose $\Iaz_k(0) = \bar{\Ia}_k$ and $\Iao_k(0) = \Iat_k(0) = 0$ to fix initial conditions on the $\Ia$s and 
$\thaz_k(0) = \bar{\tha}_k$ and $\thao_k(0) = \that_k(0) = 0$ to fix initial conditions on the $\tha$s.\\

The $\epsilon^0$ order equations are
\begin{align}
    \dt\Iaz_k &= 0   \\
    \dt\thaz_k &= \Omgaz_k, 
\end{align}
with solutions 
\begin{align}
    \Iaz_k(t) &= \bar{\Ia}_k \label{solzero1}\\
    \thaz_k(t) &= \bar{\tha}_k + \bar{\Omga}_kt, \label{solzero2}
\end{align}
where $\Omgaz_k$ and $\bar{\Omga}_k$ refer to $\Omga_k$ with only zeroeth order contribution or initial conditions respectively.
Notice that the $\epsilon$ terms in 
the shifted frequency should be included in the equations for $\thao$ and not $\thaz$, we however make this choice to keep 
all terms linear with time togheter (and thus leading in the expansion).\\
This order reproduces the dynamics of an infinite dimensional integrable system (for example infinitely many decoupled harmonic oscillators), with constant actions and angles evolving linearly with time. \\

At $\epsilon$ order the equations of motion are 
\begin{align}
    \dt\Iao_k &= 2\sum_{123} \Tint_{k123} \sqrt{\Iaz_k\Iaz_1\Iaz_2\Iaz_3} \sin(\deltheta^{k1(0)}_{23})\delta^{k1}_{23} \label{hameqone}\\
    \dt\thao_k &= \sum_{123} \Tint_{k123} \sqrt{\frac{\Iaz_1\Iaz_2\Iaz_3}{\Iaz_k}}\cos(\deltheta^{k1(0)}_{23})\delta^{k1}_{23}.
\end{align}
Here the only time dependance lies in $\Delta\theta^{(0)}$ and $\Iao_k(0) = \thao_k(0) = 0$, integrating the equations gives
\begin{align}
    \Iao_k(t) &= 2\sum_{123} \Tint_{k123} \sqrt{\Iab_k\Iab_1\Iab_2\Iab_3} \frac{\delta^{k1}_{23}}{\delOmegab^{k1}_{23}}
    \left[ \cos(\delthetab^{k1}_{23}) - \cos(\delthetab^{k1}_{23} + \delOmegab^{k1}_{23}t) \right] \label{solone1}\\
    \thao_k(t) &= \sum_{123} \Tint_{k123} \sqrt{\frac{\Iab_1\Iab_2\Iab_3}{\Iab_k}} \frac{\delta^{k1}_{23}}{\delOmegab^{k1}_{23}}
    \left[ \sin(\delthetab^{k1}_{23} + \delOmegab^{k1}_{23}t) -\sin(\delthetab^{k1}_{23}) \right]. \label{solone2}
\end{align}
Where $\delOmegab$ is defined in the same fashion as $\deltheta$. \\


We should be content with this first nontrivial result, but through the sheer power of hindsight\footnote{Developing a 
statistical theory of the system, the first nontrivial contribution comes from the $\epsilon^2$ order.}
we write also the 
$\epsilon^2$ order equations only for the action variables (there is no need to actually solve them). \\
Looking at \eqref{Hameq1} we seek to obtain an $\epsilon^2$ equation by substituting $\Ia$ and $\tha$ up to their $\epsilon$ order terms. By Taylor expanding the square root 
we obtain four terms of the form
\begin{equation}
    \sqrt{(\mathrm{x} +\epsilon \mathrm{y})\mathrm{\tilde{x}}} \underset{\epsilon \rightarrow 0}{\sim} 
    \sqrt{\mathrm{x}\mathrm{\tilde{x}}}\left( 1 + \frac{\epsilon \mathrm{y}}{2\mathrm{\tilde{x}}}\right),
\end{equation}
where, for example, $\mathrm{x} + \epsilon \mathrm{y} = \Iaz_k + \epsilon \Iao_k$ and $\mathrm{\tilde{x}} = \Iaz_1\Iaz_2\Iaz_3$.\\
There also appear terms of the form
\begin{equation}
    \sin(\mathrm{x} + \epsilon \mathrm{y}) \underset{\epsilon \rightarrow 0}{\sim} \sin(\mathrm{x}) + \epsilon \mathrm{y} \cos(\mathrm{x}),
\end{equation} 
where $\mathrm{x} = \deltheta^{k1(0)}_{23}$ and $\mathrm{y} = \deltheta^{k1(1)}_{23}$.\\
By plugging everything into \eqref{Hameq1} we first obtain
\begin{equation}
    \dt \Iat_k = 2 \sum_{123} \Tint_{k123} \sqrt{\Iaz_k\Iaz_1\Iaz_2\Iaz_3} \left[ \frac{1}{2}\left(\frac{\Iao_k}{\Iaz_k}+\frac{\Iao_1}{\Iaz_1}+
    \frac{\Iao_2}{\Iaz_2}+\frac{\Iao_3}{\Iaz_3} \right)\sin(\deltheta^{k1(0)}_{23})\delta^{k1}_{23} + 
    \deltheta^{k1(1)}_{23}\cos(\deltheta^{k1(0)}_{23}) \delta^{k1}_{23}\right],
\end{equation}
and then by using \eqref{solzero1}, \eqref{solzero2}, \eqref{solone1}, \eqref{solone2} and basic trigonometry we find
\begin{multline}
    \dt \Iat_k = 2 \sum_{123456}\Tint_{k123} \sqrt{\Iab_k\Iab_1\Iab_2\Iab_3\Iab_4\Iab_5\Iab_6} 
    \sum_{i=0}^{3}\frac{\Tint_{k123}\Tint_{i456}}{\sqrt{\Iab_i}\delOmegab^{i4}_{56}} \\
    \times\left(\sin(\delthetab^{k1}_{23} + \delOmegab^{k1}_{23}t - \sigma_i \delthetab^{i4}_{56}) 
    + \sin(\sigma_i \delthetab^{i4}_{56} + \sigma_i\delOmegab^{i4}_{56}t -\delthetab^{k1}_{23} - \delOmegab^{k1}_{23}t)  \right)\delta^{k1}_{23}\delta^{i4}_{56}, 
    \label{eqeps2}
\end{multline}
where $\sigma_i$ is equal to $+1$ if $i=0,1$ and alternatively is $-1$. When $i=0$ it represents functional dependance on $k$.\\
\subsection{Random Phase Approximation}
Having approximated the solutions to order $\epsilon$ we found ourselves with the problem of gathering 
initial conditions in infinite dimensonal systems
\footnote{Let us think of the ocean surface for example, measuring its height at a generic instant would be unfeasible.}
, we shall then renounce the deterministic approach in favour of 
a probabilistic one.\\
In general such idea is realized thorugh averaging over infinitely many realizations of the equations of motion with different initial conditions, to then extract 
average quantities more easily confrontable with experiment. In a nonlinear problem this is again highly non trivial, to simplify the endeavor we assume that
a large number of waves is present in the system, in the sense that each mode in Fourier space is highly excited. It is then reasonable to assume that, after a time
evolution proportional to the minimum value of $\frac{1}{\omega_k}$ in the range of physical interest, the phases $\tha$ would be uniformly distributed in the $\left[0,2\pi\right]$ segment 
\footnote{This is known as the random phase approximation. One shall be careful as if the original equations of motion are known to have solitonic solutions in a certain regime of k-space, 
in such case phases could be correlated and the assumption would not hold.}
. This means that whatever our initial conditions, given that $\Iab_k \neq 0$ almost everywhere and the nonlinear contribution being slower than the linear one,
we may actually assume some new initial conditions on $\theta$s drawn from the following distribution
\begin{equation}
    \left\langle f(\bar{\tha}_1 \dots \bar{\tha}_N) \right\rangle_{\thab} = 
    \int_{0}^{2\pi} P(\bar{\tha}_1 \dots \bar{\tha}_N)f(\bar{\tha}_1 \dots \bar{\tha}_N) d\bar{\tha}_1 \dots d\bar{\tha}_N 
    \hspace{3mm} \text{with} \hspace{3mm}
    P(\bar{\tha}_1 \dots \bar{\tha}_N) = \frac{1}{2\pi^{N}}
\end{equation}

Looking back at the Hamiltonian \eqref{ham4} we see that the phases do not contribute to physical quantities like the energy or the wave number, 
it is in the action variables that those observables are encoded. We have now a clear plan, to find a kinetic equation, independent of initial conditions, for
the action variables. \\
The main objective is then
\begin{equation}
    \left\langle \dt \Ia_k \right\rangle_{\thab} = \dt \left\langle \Ia_k \right\rangle_{\thab} = \left\langle \dt \Iaz_k \right\rangle_{\thab}+
    \epsilon\left\langle \dt \Iao_k \right\rangle_{\thab} + \epsilon^2\left\langle \dt \Iat_k \right\rangle_{\thab} 
    \label{kineticexp}
\end{equation}
To zeroeth order, being constant, is null. We average over the $\epsilon$ order equation \eqref{hameqone} (with subbed zeroeth order solutions)
\begin{equation}
    \left\langle \dt \Iao_k \right\rangle_{\thab} = 2\sum_{123} \Tint_{k123} \sqrt{\Iab_k\Iab_1\Iab_2\Iab_3} \left\langle\sin(\delthetab^{k1}_{23} +
    \delOmegab^{k1}_{23}t)\right\rangle_{\thab}\delta^{k1}_{23}.
\end{equation}
Making the probability distribution explicit and isolating the term depending on phases we obtain
\begin{multline}
    \left\langle\sin(\delthetab^{k1}_{23} + \delOmegab^{k1}_{23}t)\right\rangle_{\thab} = 
    \left\langle 2 \operatorname{Im}\left(e^{i\delthetab^{k1}_{23}}e^{\delOmegab^{k1}_{23}t)} \right)   \right\rangle_{\thab} = 
    \frac{1}{2\pi^4}\int_{0}^{2\pi}  e^{i\thab_k}e^{i\thab_1}e^{i\thab_2}e^{i\thab_3} d \thab_k d \thab_1 d \thab_2 d \thab_3 = 0
\end{multline}
To first order we have a trivial kinetic equation, and must then go to $\epsilon^2$ order to find nontrivial results, luckily we have already written $\Ia$'s 
Hamilton equations to second order. \\

The $\thab$ dependent part of equation \eqref{eqeps2} may be rewritten as 
\begin{equation}
    e^{+i\sigma_i\delthetab^{i4}_{56} -i \delthetab^{k1}_{23}}\left[e^{-i\delOmegab^{k1}_{23}}\left(e^{i\sigma_i\delOmegab^{i4}_{56}t}-1\right)  \right]
    \label{thetaterm}
\end{equation}
We shall focus on the $i=0$ term and extend the results to the other ones. Isolating the exponential with $\thab$ in \eqref{thetaterm} and averaging we get
\begin{equation}
    \left\langle e^{i(\thab_4 +\thab_2 + \thab_3 - \thab_5 - \thab_6 - \thab_1)} \right\rangle_{\thab}.
    \label{average}
\end{equation} 
This term is different from $0$ only if the total exponent is null. It acts as a Kroenecker's delta on the $3$ out of the $6$ sums, imposing either $k_4=k_1$ \& $k_2 = k_5$ \&
$k_3 = k_6$ or $k_4=k_1$ \& $k_2 = k_6$ \& $k_3 = k_5$\footnote{The combinations with $k_2 = k_4$ or $k_2 = k_3$ were 
excluded from the sum with the shift \eqref{shift_eq}}. \\
The full averaged $i=0$ term, with \eqref{average} enforced, is 
\begin{equation}
    4 \sum_{123} \Tint_{k123}\Tint_{k123}\Iab_1\Iab_2\Iab_3\frac{\sin(\delOmegab^{k1}_{23}t)}{\delOmegab^{k1}_{23}}\delta^{k1}_{23},
\end{equation}
where the property $\Tint_{k123} = \Tint_{k132}$ was used\footnote{In the case of complex $\Tint$ the property would hold as well with a complex conjugate on one side.}.
\\
Looking at the cases $i =1,2,3$ the only differences are:
\begin{itemize}
    \item $i=1$ $\longrightarrow$ the same as $i=0$ except for the exchange $\Iab_1 \longleftrightarrow \Iab_k$;
    \item $i=2$ $\longrightarrow$ the same as $i=0$ except for an overall minus sign;
    \item $i=3$ $\longrightarrow$ the same as $i=2$ except for  $\Iab_2 \longleftrightarrow \Iab_3$.
\end{itemize}
The final result is
\begin{equation}
    \dt \left\langle \Iat_k\right\rangle = 4 \sum_{123} \Tint^2_{k123}\Iab_k\Iab_1\Iab_2\Iab_3
    \left(\frac{1}{\Iab_k} + \frac{1}{\Iab_1} - \frac{1}{\Iab_2}- \frac{1}{\Iab_3}  \right)
    \frac{\sin(\delOmegab^{k1}_{23}t)}{\delOmegab^{k1}_{23}}\delta^{k1}_{23},
    \label{kinetic1}
\end{equation}
where it is not anymore necessary to account in the sum for trivial 
interactions, as for \\ $k_2=k \hspace{2mm}\&\hspace{2mm} k_1 = k_3$, $k_3=k \hspace{2mm}\& \hspace{2mm}k_1=k_2$ and $k_1=k_2=k_3=k$ the r.h.s is null.\\
We may ignore to this order the frequency shift as Taylor expanding the sine function shows its contribution to be of order $\epsilon^3$. \\
By using one of the possible definitions of the Dirac's delta function
\begin{equation}
    \underset{a \rightarrow \infty}{\lim} \frac{\sin(a \mathrm{x})}{\pi\mathrm{x}} = \delta(\mathrm{x}),
    \label{delta}
\end{equation}
into \eqref{kinetic1} togheter with \eqref{kineticexp} and the assumption that enough time has passed
%\footnote{More time than the fast linear time scale and smaller tha}
, finally
\begin{equation}
    \dt \left\langle \Ia_k\right\rangle = 4 \pi\epsilon^2 \sum_{123} \Tint^2_{k123}\Iab_k\Iab_1\Iab_2\Iab_3
    \left(\frac{1}{\Iab_k} + \frac{1}{\Iab_1} - \frac{1}{\Iab_2}- \frac{1}{\Iab_3}  \right)
    \delta(\delomega^{k1}_{23})\delta^{k1}_{23}.
    \label{kinetic2}
\end{equation}
There are some important remarks on this last equation. \\
The presence of the Dirac's delta defines a resonance manifold\footnote{even if at this discrete stage $\delomega$ is not a function over $\mathbb{R}$ yet,
 and should be argued to be densely valued around $0$ in the continuum limit} of the Fourier modes $k_1, k_2$ and $k_3$ interacting with $k$, 
 showing that only resonant term contribute to the net interaction between different modes\footnote{
    This can be seen as an a posteriori justification for the elimination of nonresonant three wave interaction terms.
}. \\
Based on difference preferences one could define a nonlinear time $\tau = t \epsilon^2$, as in \cite{Onorato2020}, and absorb the 
$\epsilon^2$ term into the time derivative or just include it again into $\Tint$, we opt for the latter and will not write it explicitely in the future. \\
The equation can be readily extended to the case $\Tint \in \mathbb{C}$ through the substitution $\Tint^2 \rightarrow |\Tint|^2$.\\
We solved the problem of initial conditions on the phases but not yet on the action variables and we are still dealing with infinite sums, since we defined our system
in a finite coordinate space. The next section builds on this. 

\subsection{Statistics over the actions and the thermodynamic limit}
We now take the thermodynamic (continuum) limit ($L \rightarrow \infty$) to turn our set of infinitely many coupled equation into an integral one, easier to approach analytically.
In Fourier space the limit takes the form 
\begin{equation}
    \vec{\Delta k} = \frac{2\pi}{L} \rightarrow 0 \hspace{10mm} \Lambda^* \rightarrow \mathbb{R}^d
\end{equation}
and we define 
\begin{equation}
    \tilde{\Ia}_k=\frac{\Ia_k}{(\Delta k)^d} \rightarrow \tilde{\Ia}(k) \hspace{10mm} \sum_k (\Delta k)^d \rightarrow \int d^d k \hspace{10mm} 
    \frac{(\delta^{k1}_{23})^d}{(\Delta k)^d} \rightarrow \delta^d(k + k_1 - k_2 - k_3).
\end{equation}
The action variables and interaction coefficents become functions of the now continuous coordinates of Fourier space, the sums turn into integrals and the Kroenecker's deltas become a
Dirac's delta. Equation \eqref{kinetic2}, suppressing again all dimensional indexes, becomes
\begin{multline}
    \pt \left\langle \tilde{\Ia}(k)\right\rangle = 4 \pi\int dk_1 dk_2 dk_3 \\
    \Tint^2(k, k_1, k_2, k_3)\Iab(k)\Iab(k_1)\Iab(k_2)\Iab(k_3)
    \left(\frac{1}{\Iab(k)} + \frac{1}{\Iab(k_1)} - \frac{1}{\Iab(k_2)}- \frac{1}{\Iab(k_3)}  \right)
    \delta(\delomega^{k1}_{23})\delta(k + k_1 - k_2 - k_3).
    \label{kinetic3}
\end{multline}
In performing the limit we must be careful to require that 
\begin{equation}
\underset{\epsilon, \Delta k \rightarrow 0}{\lim} \frac{(\Delta k)^d}{\epsilon} = 0 ,
\end{equation}
as otherwise we could not use definition \eqref{delta} to extract a delta from \eqref{kinetic1}. \\
In the following we will recover the previous discrete notation for compactness, while still working in the continous limit.\\

Our equation is still deterministic regarding the action function. We shall assume a stochastic distribution on the inital data defining 
the mean value\footnote{Working with $a_k$ variables $n_k$ would be the second moment of the distribution, not the first.}
of $\Ia_k$ respect both distribution at a generic time
\begin{equation}
    n(k,t) = n_k = \left\langle \Ia_k \right\rangle_{\Iab, \thab},
\end{equation}     
and requiring them to be uncorrelated at $t=0$ \\
\begin{equation}
    \left\langle \Iab_i \Iab_j \Iab_k\right\rangle_{\Iab} = 
    \left\langle\Iab_i \right\rangle_{\Iab} \left\langle \Iab_j \right\rangle_{\Iab} \left\langle \Iab_k \right\rangle_{\Iab}
    = \bar{n}_i \bar{n}_j \bar{n}_k \hspace{5mm} \text{for} \hspace{5mm} i\neq j\neq k, 
\end{equation}
basically a gaussian distribution on initial conditions \hl{JUSTIFY THIS INTUITIVELY}. Based on dimensional analysis we call 
$n_k$ the wave action density of the system\footnote{$\Nwav = \int n_k dk$ has the units of an action.}.\\
Averaging over \eqref{kinetic3} we obtain
\begin{equation}
    \pt n_k = 4 \pi\int dk_1 dk_2 dk_3 \\
    \Tint^2_{k123}\nb_k\nb_1\nb_2\nb_3
    \left(\frac{1}{\nb_k} + \frac{1}{\nb_1} - \frac{1}{\nb_2}- \frac{1}{\nb_3}  \right)
    \delta(\delomega^{k1}_{23})\delta^{k1}_{23}.
    \label{kinetic4}
\end{equation}
We find ourselves with an equation describing the time evolution of the average action value per Fourier mode, given an initial distribution.
Unfortunately this equation holds now for a small time after $t=0$, 
since it does not account for successive interactions between different waves. We need some form of closure. \hl{DOES IT MAKE SENSE?, THINK AGAIN. LOOK ON ZAKH HOW HE JUSTIFY THIS}\\ 
We make a last crucial assumption, time evolution does not spoil at successive instants the random phase and amplitude 
assumptions on inital conditions. Intuitively we can think of the separation of time scales on which the linear and nonlinear term act as allowing for 
the linear term to keep introducing chaos into the system. \\
This assumption let us subsitute $\nb_k \rightarrow n(k,t) = n_k$, giving us a self-consistent integro-differential equation for the evolution of $n_k$, the celebrated
Wave Kinetic Equation (WKE)
\begin{equation}
    \pt n_k = 4 \pi\int dk_1 dk_2 dk_3 \\
    \Tint^2_{k123}n_kn_1n_2n_3
    \left(\frac{1}{n_k} + \frac{1}{n_1} - \frac{1}{n_2}- \frac{1}{n_3}  \right)
    \delta(\delomega^{k1}_{23})\delta^{k1}_{23}.
    \label{kinetic}
\end{equation}
Given that we are assuming gaussian statistics, but at the same time letting the mean values of the infinite distributions at every point k 
vary with time depending on each other values, it is customary to call the approximation quasi-gaussian. The l.h.s. is often called the collisional integral, again 
for the reason that it quantifies the interactions among different modes, and represented through the symbol $\coll$.

\subsection{Main properties of the kinetic equation}
\hl{CONSERVATION LAWS (E, N ,GENERICA RHO), IRREVERSIBILITÀ}\\
The Microscopic Hamiltonian \eqref{ham4} conserves energy, momentum and wave number (or wave action) as a consequence of its invariance under time translation,
space translation and phase shifting\footnote{A three wave interaction would not conserve the wave action} respectively. \\
A first question would be which of said conserved quantities are inherited by the WKE.\\
Let us start with energy, we separe the linear and nonlinear terms in \eqref{ham4}
\begin{equation}
    \Ham = \Ham_2 + \Ham_{\text{int}}.
\end{equation}
The derivation of the kinetic equation was based off the assumption that $\Ham_{\text{int}} \ll \Ham_{2}$ and the obtained equation clearly shows that at $\epsilon^2$ 
order the only role of $\Ham_{\text{int}}$ is to redistibute energy amidst modes. This is enough to hypothesize that the conserved quantity may be the average linear energy
\begin{equation}
    \Ewav = \left\langle \Ham_2 \right\rangle_{\Iab,\thab} = \int \omega_k n_k dk.
\end{equation}
Its conservation may be easily checked by utilizing \eqref{kinetic} 
\begin{align}
    &\dt \Ewav = \int \omega_k \pt n_k dk = \int \omega_k \coll dk = \label{encons}\\
    4 &\pi\int  dk dk_1 dk_2 dk_3 
    \Tint^2_{k123}n_kn_1n_2n_3 \omega_k
    \left(\frac{1}{n_k} + \frac{1}{n_1} - \frac{1}{n_2}- \frac{1}{n_3}  \right)
    \delta(\delomega^{k1}_{23})\delta^{k1}_{23} = \\
    &\pi\int  dk dk_1 dk_2 dk_3 
    \Tint^2_{k123}n_kn_1n_2n_3\left(\omega_k + \omega_1 - \omega_2 - \omega_3\right)
    \left(\frac{\omega_k}{n_k} + \frac{\omega_1}{n_1} - \frac{\omega_k}{n_2}- \frac{\omega_k}{n_3}  \right)
    \delta(\delomega^{k1}_{23})\delta^{k1}_{23} = 0.    
\end{align}
The simmetry properties of $\Tint$ and the rest of the integrand were exploited to properly rename the dummy integrated variables. The delta function over frequencies 
allows to determine that the integral is null. It is crucial to remember that the usage of the delta function is allowed only if the integral converges, such condition
must be checked case by case and thus conservation of energy is not to be taken for granted\footnote{See \cite{Zakharov} for a more detailed analysis.}. \\
Defining the energy density as $\Edens_k = \omega_k n_k$, we cast \eqref{encons} as a continuity equation
\begin{align}
    &\pt \Edens_k + \vec{\nabla}\cdot\vec{\Pflux} = 0 \label{Econtinuity}\\
    &\vec{\nabla}\cdot\vec{\Pflux} = -\omega_k \coll, 
\end{align}
where $\Pflux$ is the energy flux and the vector signs were made e to xplicit to avoid ambiguity with future notation. \\

We now move on to wave action conservation, given the original simmetry it is reasonable to assume that the new conserved quantity is 
\begin{equation}
    \Nwav = \int n_k dk.
\end{equation}
 
\subsection{Equilibrium stationary state}

-Equilibrium solutions and comments 

\subsection{Out of equilibrium stationary states}
\hl{QUESTA PARTE VA FATTA BENE, VA PROGRAMMATA COORDINATA CON MMT!!}
-Out of Equilibrium solutions and comments, argomento fjortoft, argomento zakharov, misto dei due con commenti su forzante e dissipazione 
ed esempi fisici, ragionamenti dimensionali su forme approssimate di lambda e omega e forme generiche di esponenti e finestra di località, definizioni costanti di KZ e commenti su cnvergenza e 
trasformata zakh (forse è il caso di presentarla solo in MMT? bouch) un pò di foto di cascate per sistemi fisici NON MMT

\subsection{Convergence issues (MAYBE)}